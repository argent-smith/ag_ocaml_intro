\documentclass[12pt,aspectratio=169,ignorenonframetext,hyperref={pdftex,unicode},xcolor=dvipsnames]{beamer}
\usepackage[english,russian]{babel}
\usepackage[utf8x]{inputenc}
\usepackage{textcomp}
\usepackage{cmap}
\usepackage{mathptmx}
\usepackage[T2A]{fontenc}
\usepackage{multicol}
\usepackage{pgfpages}
\usepackage{amsmath}
\usepackage{ulem}
%\usepackage{hhline}
%\usepackage{alltt}
%\usepackage{syntax}
\usepackage{nicefrac}
\usepackage[absolute,overlay]{textpos}
\usepackage{fancyvrb}
\usepackage{tikz}
\usetikzlibrary{shapes,arrows,positioning,calc,chains,scopes,trees}
\usetikzlibrary{arrows.meta,matrix,tikzmark}
%\usepackage{mathpartir}
%\usepackage{colonequals}
%\usepackage{tabularx}
%\usepackage{ifmtarg}
\usepackage{paratype}
\usepackage[cache=false]{minted}
\usepackage{fvextra}
\usepackage{upquote}
\usepackage{mdwlist}
\usepackage{tcolorbox}
\usepackage[thicklines]{cancel}
\usepackage{tabularx}
\usepackage{bussproofs}

\renewcommand{\CancelColor}{\color{red}}

\newminted{ocaml}{linenos=true}
\newmintinline[caml]{ocaml}{}
\newminted{text}{}
\newmintinline[txt]{text}{}
\newminted{scheme}{}
\newmintinline[scheme]{scheme}{}
% \newminted{sql}{}
% \newminted{java}{}
% \newmintinline[jj]{java}{}

\definecolor{gray}{gray}{0.5}
\definecolor{green}{rgb}{0,0.5,0}


\newcommand{\slidegraphics}[2]{
\begin{center}
 \includegraphics[#1,keepaspectratio]{./images/#2}
\end{center}}

\usetheme{metropolis}           % Use metropolis theme
\metroset{numbering=counter}
\metroset{subsectionpage=progressbar}
\metroset{block=fill}

% \usetheme{Warsaw}
% \useoutertheme{infolines}
% \usecolortheme{crane}

% \setbeamertemplate{footline}
% {%
% \leavevmode%
% \hbox{%

% \begin{beamercolorbox}[wd=.4\paperwidth,ht=2.5ex,dp=1.125ex,right]{author
% in head/foot}%
% \usebeamerfont{title in head/foot}\insertshortauthor\
% (\insertshortinstitute)\hspace{.3cm}
% \end{beamercolorbox}%

% \begin{beamercolorbox}[wd=.5\paperwidth,ht=2.5ex,dp=1.125ex,center]{title
% in head/foot}%
% \usebeamerfont{author in head/foot}\hspace{.3cm}\insertshorttitle
% \end{beamercolorbox}%

% \begin{beamercolorbox}[wd=.1\paperwidth,ht=2.5ex,dp=1.125ex,center]{author
% in head/foot}
% \usebeamerfont{author in
% head/foot}\hspace{.3cm}\insertframenumber/\inserttotalframenumber\hspace{.3cm}
% \end{beamercolorbox}%
% }%
% \vskip0pt%
% }
% \setbeamertemplate{navigation symbols}{}

% \AtBeginSection[]
% {
% \begin{frame}<*>
% \frametitle{Содержание}
% \tableofcontents[currentsection,subsectionstyle=show/show/hide]
% \end{frame}
% }

% \AtBeginSubsection[]
% {
% \begin{frame}<*>
% \frametitle{Содержание}
% \tableofcontents[currentsection,subsectionstyle=show/shaded/hide]
% \end{frame}
% }

\institute{}


\author{Павел Аргентов}

\newcommand{\graybg}[1]{\colorbox{gray}{\mbox{#1}}}
\renewcommand{\L}{\ensuremath{\lambda}}
\newcommand{\LL}{\ensuremath{\Lambda}}

% %%%
% %%% НОВАЯ ВЕРСТКА ТАБЛИЦ
% %%%
% %%% сначала идёт параграф с заголовком...
% \newcommand{\displayNew}[3]{#1

% %%% ... потом верхняя рамка
% %\rule[-1ex]{.4pt}{1ex}\rule{\tapltablewidth}{.4pt}\rule[-1ex]{.4pt}{1ex}\smallskip

% %%% ... потом две minipage, стоящие рядом через черточку
% \begin{minipage}[t]{.5\tapltablewidth}
% #2
% \end{minipage}
% \vline
% \begin{minipage}[t]{.5\tapltablewidth}
% #3
% \end{minipage}

% %%% ... потом нижняя рамка
% %\rule{.4pt}{1ex}\rule{\tapltablewidth}{.4pt}\rule{.4pt}{1ex}\par
% }

% \makeatletter
% \newcommand{\tablesection}[2]{\textit{#1}\@ifmtarg{#2}{\relax}{\hfill\fbox{#2}}}
% \makeatother

% % \newlength{\tapltablewidth}
% % \setlength{\tapltablewidth}{\paperwidth}
% % \addtolength\tapltablewidth{-10mm}

% \newcommand{\code}[1]{\texttt{#1}}
% \newcommand{\ruleid}[1]{\mbox{\textsc{#1}}}
% \newcommand{\ruletag}[1]{\tag{\textsc{#1}}\index{правило!\textsc{#1}}}
% \newcommand{\function}[1]{\mbox{\textit{#1}\,}}

\newcommand{\bt}{$^\backprime$}


\def\PYGdefaultZsq{\textquotesingle}

\makeatletter
\let\@sverbatim\@verbatim
\def\@verbatim{\@sverbatim \verbatimwithtick}
{\catcode``=13 \gdef\verbatimwithtick{\chardef`=18 }}
\makeatother


\newcommand{\cfbox}[2]{%
    \colorlet{currentcolor}{.}%
    {\color{#1}%
    \fbox{\color{currentcolor}#2}}%
}

\newcommand{\cffrac}[3]{%
  \cfbox{#1}{$\frac{#2}{#3}$}%
}

\newcommand{\wffrac}[2]{\cffrac{white}{#1}{#2}}
\newcommand{\rffrac}[2]{\cffrac{red}{#1}{#2}}

\newenvironment{mycbox}{\begin{tcolorbox}[boxrule=0.2mm,colback=gray!30]\begin{flushleft}}{\end{flushleft}\end{tcolorbox}}


\newcommand{\central}[1]{\begin{center}\Huge\color{green} \textbf{#1}\end{center}}



\newcommand{\screenshot}[1]{
\begin{center}
\includegraphics[width=12cm,keepaspectratio]{./images/#1}
\end{center}
}

\newcommand{\screenshotw}[2]{
\begin{center}
\includegraphics[width=#1,keepaspectratio]{./images/#2}
\end{center}
}

\newenvironment{idris}
  {\large\Verbatim[commandchars=\\\{\}]}
  {\endVerbatim}

\newenvironment{idrisNormal}
  {\Verbatim[commandchars=\\\{\}]}
  {\endVerbatim}


\newcommand{\IdrisData}[1]{\textcolor{red}{#1}}
\newcommand{\IdrisType}[1]{\textcolor{blue}{#1}}
\newcommand{\IdrisBound}[1]{\textcolor{magenta}{#1}}
\newcommand{\IdrisFunction}[1]{\textcolor{green}{#1}}
\newcommand{\IdrisKeyword}[1]{{\underline{#1}}}
\newcommand{\IdrisImplicit}[1]{{\itshape \IdrisBound{#1}}}

\newcommand{\ruleCat}[1]{\ensuremath{\mathrm{-#1}}}

\newcommand{\CC}{C\nolinebreak\hspace{-.05em}\raisebox{.4ex}{\tiny\bf +}\nolinebreak\hspace{-.10em}\raisebox{.4ex}{\tiny\bf +}}
\def\CC{{C\nolinebreak[4]\hspace{-.05em}\raisebox{.4ex}{\tiny\bf ++}}}

\newcommand{\lang}[1]{\begin{flushright}\textit{\color{gray}#1}\end{flushright}}
